\documentclass[letterpaper,11pt]{article}

\usepackage{latexsym}
\usepackage[empty]{fullpage}
\usepackage{titlesec}
\usepackage{marvosym}
\usepackage[usenames,dvipsnames]{color}
\usepackage{verbatim}
\usepackage{enumitem}
\usepackage[hidelinks]{hyperref}
\usepackage{fancyhdr}
\usepackage[english]{babel}
\usepackage{tabularx}
\usepackage{fontawesome5}
\usepackage{multicol}
\setlength{\multicolsep}{-3.0pt}
\setlength{\columnsep}{-1pt}
\input{glyphtounicode}


%----------FONT OPTIONS----------
% sans-serif
% \usepackage[sfdefault]{FiraSans}
% \usepackage[sfdefault]{roboto}
% \usepackage[sfdefault]{noto-sans}
% \usepackage[default]{sourcesanspro}

% serif
% \usepackage{CormorantGaramond}
% \usepackage{charter}


\pagestyle{fancy}
\fancyhf{} % clear all header and footer fields
\fancyfoot{}
\renewcommand{\headrulewidth}{0pt}
\renewcommand{\footrulewidth}{0pt}

% Adjust margins
\addtolength{\oddsidemargin}{-0.6in}
\addtolength{\evensidemargin}{-0.5in}
\addtolength{\textwidth}{1.19in}
\addtolength{\topmargin}{-.7in}
\addtolength{\textheight}{1.4in}

\urlstyle{same}

\raggedbottom
\raggedright
\setlength{\tabcolsep}{0in}

% Sections formatting
\titleformat{\section}{
  \vspace{-4pt}\scshape\raggedright\large\bfseries
}{}{0em}{}[\color{black}\titlerule \vspace{-5pt}]

% Ensure that generate pdf is machine readable/ATS parsable
\pdfgentounicode=1

%-------------------------
% Custom commands
\newcommand{\resumeItem}[1]{
  \item\small{
    {#1 \vspace{-2pt}}
  }
}

\newcommand{\classesList}[4]{
    \item\small{
        {#1 #2 #3 #4 \vspace{-2pt}}
  }
}

\newcommand{\resumeSubheading}[4]{
  \vspace{-2pt}\item
    \begin{tabular*}{1.0\textwidth}[t]{l@{\extracolsep{\fill}}r}
      \textbf{#1} & \textbf{\small #2} \\
      \textit{\small#3} & \textit{\small #4} \\
    \end{tabular*}\vspace{-7pt}
}

\newcommand{\resumeSubSubheading}[2]{
    \item
    \begin{tabular*}{0.97\textwidth}{l@{\extracolsep{\fill}}r}
      \textit{\small#1} & \textit{\small #2} \\
    \end{tabular*}\vspace{-7pt}
}

\newcommand{\resumeProjectHeading}[2]{
    \item
    \begin{tabular*}{1.001\textwidth}{l@{\extracolsep{\fill}}r}
      \small#1 & \textbf{\small #2}\\
    \end{tabular*}\vspace{-7pt}
}

\newcommand{\resumeSubItem}[1]{\resumeItem{#1}\vspace{-4pt}}

\renewcommand\labelitemi{$\vcenter{\hbox{\tiny$\bullet$}}$}
\renewcommand\labelitemii{$\vcenter{\hbox{\tiny$\bullet$}}$}

\newcommand{\resumeSubHeadingListStart}{\begin{itemize}[leftmargin=0.0in, label={}]}
\newcommand{\resumeSubHeadingListEnd}{\end{itemize}}
\newcommand{\resumeItemListStart}{\begin{itemize}}
\newcommand{\resumeItemListEnd}{\end{itemize}\vspace{-5pt}}

%-------------------------------------------
%%%%%%  RESUME STARTS HERE  %%%%%%%%%%%%%%%%%%%%%%%%%%%%


\begin{document}

%----------HEADING----------

\begin{center}
    \textbf{\Huge \scshape Fernando Berrospi} \\ 
    \vspace{1pt}
    Lima, Peru \\ 
    \vspace{1pt}
    \small
    \raisebox{-0.1\height}\ 989-077-376 ~
    \href{mailto:fberrosp@gmail.com}{\raisebox{-0.2\height}\  fberrosp@gmail.com} ~ 
    \href{https://www.linkedin.com/in/fberrosp/}{\raisebox{-0.2\height}\ linkedin.com/in/fberrosp}  ~
    \href{https://fberrosp.github.io}{\raisebox{-0.2\height}\ fberrosp.github.io} ~
    \href{https://github.com/fberrosp}{\raisebox{-0.2\height}\ github.com/fberrosp}
    \vspace{-8pt}
\end{center}


%-----------EXPERIENCE-----------
\section{Experiencia}
  \resumeSubHeadingListStart

    \resumeSubheading
      {MS4M}{Agosto 2021 -- Septiembre 2022}
      {Ingeniero de Software}{Lima, Peru}
      \resumeItemListStart
        \resumeItem{Participación en la creación de un modelo de machine learning de detección de puntos de referencia faciales basado en la arquitectura MobileNetV2 para detectar rasgos faciales clave que permitan detectar la fatiga de los conductores en las minas sudamericanas.}
        \resumeItem{Colaboración en la revisión de las métricas de detección de fatiga, como el porcentaje de ojos cerrados durante un periodo de tiempo (PERCLOS), la relación de aspecto de los ojos (EAR) y la relación de aspecto de la boca (MAR), logrando una mejora en la precisión del modelo del 27,91\% al 87,10\%.}
        \resumeItem{Construcción de un algoritmo de optimización para identificar el modelo con los mejores hiper parámetros como el número apropiado de épocas y la tasa de aprendizaje para obtener los resultados óptimos de detección de parpadeos y bostezos.}
      \resumeItemListEnd

    \resumeSubheading
      {CDC Gold}{Enero 2019 -- Julio 2019}
      {Desarrollador de Software Junior}{La Libertad, Peru}
      \resumeItemListStart
        \resumeItem{Desarrollo de un algoritmo de k-means clustering en R para identificar los retrasos de los camiones de transporte en las instalaciones mineras.}
        \resumeItem{Diseño un algoritmo de seguimiento para rastrear las rutas tomadas por los camiones cisterna de agua, ahorrando a la empresa más de \$3000 al mes.}
        \resumeItem{Mejora de la eficiencia de la programación de los camiones mediante la automatización del proceso utilizando un script en R, optimizando el processo de programación en un 85\%.}
        \resumeItem{Supervisión de un equipo para diseñar e implantar un innovador sistema de riego, ahorrando a la empresa más de \$5000 dólares al mes.}
    \resumeItemListEnd
    
  \resumeSubHeadingListEnd
\vspace{-16pt}


%-----------PROJECTS-----------
\section{Proyectos}
    \vspace{-5pt}
    \resumeSubHeadingListStart

      \resumeProjectHeading
          {\textbf{Sistema de Gestión de Tickets} $|$ \emph{Javascript, Firebase, Bootstrap}}{Diciembre 2022}
          \resumeItemListStart
            \resumeItem{Diseño una aplicación web CRUD usando Javascript como frontend y Firebase como backend como servicio.}
            \resumeItem{Seguimiento del patrón arquitectónico Modelo, Vista, Controlador (MVC) para el diseño de la infraestructura del sistema.}
            \resumeItem{Desarrollo de la lógica del sistema utilizando Programación Orientada a Objetos (POO) para mejorar la escalabilidad y el mantenimiento del código.}
            \resumeItem{Implementado de funciones de roles y de gestión de equipos para administradores para lograr una gestión de proyectos de éxito.}
            \resumeItem{Demo: \href{https://tinyurl.com/y7xu5xdh}{https://tinyurl.com/y7xu5xdh}}
          \resumeItemListEnd
          \vspace{-13pt}

      \resumeProjectHeading
          {\textbf{Formula 1 – Análisis de Grand Prix} $|$ \emph{Python, Pandas, Seaborn}}{Septiembre 2022}
          \resumeItemListStart
            \resumeItem{Desarrollo de un informe de Análisis Exploratorio de Datos (AED) para determinar las variaciones del tiempo de vuelta en Fórmula 1 a lo largo del tiempo.}
            \resumeItem{Conducción de un exhaustivo proceso de limpieza de datos clasificando las variables nulas, realizando imputación de datos y codificando los tipos de datos categóricos en numéricos utilizando las bibliotecas de Python Pandas y Seaborn.}
            \resumeItem{Realización de análisis univariable y multivariable para determinar variables relevantes y detectar posibles correlaciones.}
            \resumeItem{Demo: \href{https://tinyurl.com/ye24ut5y}{https://tinyurl.com/ye24ut5y}}
            \resumeItemListEnd

    \resumeSubHeadingListEnd
\vspace{-15pt}


%-----------PUBLICATIONS-----------
\section{Publicaciones}
    \vspace{-5pt}
    \resumeSubHeadingListStart

      \resumeProjectHeading
          {\textbf{INTERCON 2022} $|$ \emph{IEEE}}{Septiembre 2022}
          \resumeItemListStart
            \resumeItem{A. Martinez, F. Berrospi, V. Porras and M. Portocarrero, "Using facial landmarks to detect driver fatigue," 2022 IEEE XXIX International Conference on Electronics, Electrical Engineering and Computing (INTERCON), 2022, pp. 1-4, doi: 10.1109/ INTERCON55795.2022.9870046.}
          \resumeItemListEnd

    \resumeSubHeadingListEnd
\vspace{-15pt}

%-----------EDUCATION-----------
\section{Educación}
  \resumeSubHeadingListStart
    \resumeSubheading
      {Purdue University}{Diciembre 2019}
      {Bachiller en Ciencias en Ingeniería Industrial}{West Lafayette, IN}
  \resumeSubHeadingListEnd


%-----------SKILLS-----------
%Programming Languages - Python, Javascript, R 
%Tools - Git, (MongoDB), Firebase, LaTeX, Anaconda, HTML, CSS, Microsoft Power BI
%Frameworks - (Django), (React.js), (Express.js), TensorFlow, PyTorch, OpenCV, Bootstrap
\section{Habilidades Técnicas}
 \begin{itemize}[leftmargin=0.15in, label={}]
    \small{\item{
     \textbf{Lenguajes de Programación}{: Python, Javascript, R} \\
     \textbf{Herramientas}{: Git, Firebase, LaTeX, Anaconda, HTML, CSS, Microsoft Power BI} \\
     \textbf{Frameworks}{: TensorFlow, PyTorch, OpenCV, Bootstrap} \\
    }}
 \end{itemize}
 \vspace{-16pt}

\end{document}
