%!TEX program = lualatex
\documentclass[letterpaper,11pt]{article}

\usepackage{latexsym}
\usepackage[empty]{fullpage}
\usepackage{titlesec}
\usepackage{marvosym}
\usepackage[usenames,dvipsnames]{color}
\usepackage{verbatim}
\usepackage{enumitem}
\usepackage[hidelinks]{hyperref}
\usepackage{fancyhdr}
\usepackage[english]{babel}
\usepackage{tabularx}
\usepackage{multicol}
\setlength{\multicolsep}{-3.0pt}
\setlength{\columnsep}{-1pt}
\input{glyphtounicode}


%----------FONT OPTIONS----------
% sans-serif
%\renewcommand{\familydefault}{\sfdefault}
\usepackage[default,scale=0.90]{opensans}
\usepackage[T1]{fontenc}

\pagestyle{fancy}
\fancyhf{} % clear all header and footer fields
\fancyfoot{}
\renewcommand{\headrulewidth}{0pt}
\renewcommand{\footrulewidth}{0pt}

% Adjust margins
\addtolength{\oddsidemargin}{-0.6in}
\addtolength{\evensidemargin}{-0.5in}
\addtolength{\textwidth}{1.19in}
\addtolength{\topmargin}{-.7in}
\addtolength{\textheight}{1.4in}

\urlstyle{same}

\raggedbottom
\raggedright
\setlength{\tabcolsep}{0in}

% Sections formatting
\titleformat{\section}{
  \vspace{-4pt}\scshape\raggedright\large\bfseries
}{}{0em}{}[\color{black}\titlerule \vspace{-5pt}]

% Ensure that generate pdf is machine readable/ATS parsable
\pdfgentounicode=1

%-------------------------
% Custom commands
\newcommand{\resumeItem}[1]{
  \item\small{
    {#1 \vspace{-2pt}}
  }
}

\newcommand{\classesList}[4]{
    \item\small{
        {#1 #2 #3 #4 \vspace{-2pt}}
  }
}

\newcommand{\resumeSubheading}[4]{
  \vspace{-2pt}\item
    \begin{tabular*}{1.0\textwidth}[t]{l@{\extracolsep{\fill}}r}
      \textbf{#1} & \textbf{\small #2} \\
      \textit{\small#3} & \textit{\small #4} \\
    \end{tabular*}\vspace{-7pt}
}

\newcommand{\resumeSubSubheading}[2]{
    \item
    \begin{tabular*}{0.97\textwidth}{l@{\extracolsep{\fill}}r}
      \textit{\small#1} & \textit{\small #2} \\
    \end{tabular*}\vspace{-7pt}
}

\newcommand{\resumeProjectHeading}[2]{
    \item
    \begin{tabular*}{1.001\textwidth}{l@{\extracolsep{\fill}}r}
      \small#1 & \textbf{\small #2}\\
    \end{tabular*}\vspace{-7pt}
}

\newcommand{\resumeSubItem}[1]{\resumeItem{#1}\vspace{-4pt}}

\renewcommand\labelitemi{$\vcenter{\hbox{\tiny$\bullet$}}$}
\renewcommand\labelitemii{$\vcenter{\hbox{\tiny$\bullet$}}$}

\newcommand{\resumeSubHeadingListStart}{\begin{itemize}[leftmargin=0.0in, label={}]}
\newcommand{\resumeSubHeadingListEnd}{\end{itemize}}
\newcommand{\resumeItemListStart}{\begin{itemize}}
\newcommand{\resumeItemListEnd}{\end{itemize}\vspace{-5pt}}

%-------------------------------------------
%%%%%%  RESUME STARTS HERE  %%%%%%%%%%%%%%%%%%%%%%%%%%%%


\begin{document}

%----------HEADING----------

\begin{center}
    \textbf{\Huge \scshape Fernando Berrospi} \\ \vspace{1pt}
    U.S. Green Card holder \\ \vspace{1pt}
    \small
    \raisebox{-0.1\height}\ (312) 442-0843 ~
    \href{mailto:fberrosp@gmail.com}{\raisebox{-0.2\height}\  fberrosp@gmail.com} ~ 
    \href{https://www.linkedin.com/in/fberrosp/}{\raisebox{-0.2\height}\ linkedin.com/in/fberrosp}  ~
    \href{https://fberrosp.github.io}{\raisebox{-0.2\height}\ fberrosp.github.io} ~
    \href{https://github.com/fberrosp}{\raisebox{-0.2\height}\ github.com/fberrosp}
    \vspace{-8pt}
\end{center}

%-----------SKILLS-----------
\section{Skills}
 \begin{itemize}[leftmargin=0.15in, label={}]
    \small{\item{
     \textbf{Programming Languages}{: JavaScript, Python, R} \\
     \textbf{Tools}{: React.js, Git, MongoDB, PostgreSQL, Firebase, Jira, HTML, CSS, Microsoft Power BI} \\
     \textbf{Frameworks}{: Next.js, Express.js, Django, Tailwind CSS, Bootstrap, TensorFlow, PyTorch, OpenCV} \\
    }}
 \end{itemize}
 \vspace{-15pt}

%-----------EXPERIENCE-----------
\section{Experience}
  \resumeSubHeadingListStart

  \resumeSubheading
  {PROEMISA}{January 2023 -- Present}
  {Frontend Developer $|$ React.js, Tailwind CSS, Next.js, Git}{Lima, Peru}
  \resumeItemListStart
    \resumeItem{Improved user experience by optimizing website performance, cross-browser compatibility, and mobile responsiveness, resulting in a 37\% increase in page load speed.}
    \resumeItem{Streamlined website development process by creating a comprehensive design system in Figma, including reusable components and style guides, resulting in increasing development efficiency and a reduction in design errors.}
    \resumeItem{Enhanced website visibility and user engagement by implementing SEO best practices and Google Analytics, resulting in a 14\% increase in organic traffic and a 22\% improvement in user retention rate.}
  \resumeItemListEnd

  \resumeSubheading
    {MS4M}{August 2021 -- September 2022}
    {Software Engineer $|$ Python, PostgreSQL, Jira, Power BI, Git, TensorFlow, Open CV}{Lima, Peru}
    \resumeItemListStart
      \resumeItem{Contributed to developing a machine learning facial landmarks detection model using MobileNetV2 architecture for fatigue detection in South American haul truck drivers, lowering the number of fatigue-related accidents.}
      \resumeItem{Constructed an optimization algorithm to identify the model with the best hyperparameters, resulting in a 75.82\% improvement in blink and yawn detection accuracy.}
      \resumeItem{Analyzed fatigue detection metrics and KPI data using Jira Query Language and Power BI to identify potential improvements and bottlenecks in the software development process, resulting in a 15\% increase in team productivity.}
    \resumeItemListEnd

  \resumeSubheading
    {CDC Gold}{January 2019 -- July 2019}
    {Junior Software Developer $|$ R, ggplot2, Tidyr}{La Libertad, Peru}
    \resumeItemListStart
      \resumeItem{Developed and implemented a k-means clustering algorithm in R to analyze haul truck delays in mining facilities, resulting in a 25\% reduction in delays and saving the company over \$50,000 per month.}
      \resumeItem{Designed and executed a tracking algorithm in R programming language to optimize the routes taken by water tank trucks, saving the company over \$3000 per truck, per month.}
      \resumeItem{Streamlined haul truck scheduling efficiency by automating the process using an R script, reducing the time spent on scheduling by 87\% and improving on-time delivery performance by 17\%.}
  \resumeItemListEnd
    
  \resumeSubHeadingListEnd
\vspace{-15pt}


%-----------PROJECTS-----------
\section{Projects}
    \vspace{-5pt}
    \resumeSubHeadingListStart

      \resumeProjectHeading
          {\textbf{Project Management App} $|$ \href{https://github.com/fberrosp/PMApp}{github.com/fberrosp/PMApp} $|$ \emph{MongoDB, Express.js, React.js, Node.js}}{December 2022}
          \resumeItemListStart
            \resumeItem{Developed a full-stack project management web application utilizing MERN stack and Firebase authentication, following industry best practices for creating, deploying, and documenting the application.}
            \resumeItem{Successfully implemented CRUD operations for creating, reading, updating, and deleting projects, managing users, assigning projects to users, handling comments and logs, achieving a 95\% user satisfaction rate based on user feedback.}
            \resumeItem{Followed RESTful design principles to build APIs, improving the scalability and maintainability of the application and reducing server response times by 33\%.}

          \resumeItemListEnd
          \vspace{-15pt}

      \resumeProjectHeading
          {\textbf{Formula 1 App} $|$ \href{https://github.com/fberrosp/F1App}{github.com/fberrosp/F1App} $|$ \emph{Django, Plotly, PostgreSQL, Pandas, Redis}}{September 2022}
          \resumeItemListStart
            \resumeItem{Developed a full-stack Formula 1 data science web application using Django, resulting in a responsive and user-friendly platform that visualizes historical Formula 1 data through interactive graphs and statistics.}
            \resumeItem{Optimized response time and data availability by implementing Redis caching and PostgreSQL, reducing response time by 10\% and ensuring users always had access to up-to-date data.}
            \resumeItem{ Enhanced user engagement and satisfaction by developing user-friendly interfaces and incorporating interactive graphs using Plotly, resulting in a 20\% increase in user engagement.}
            \resumeItemListEnd

    \resumeSubHeadingListEnd
\vspace{-15pt}


%-----------PUBLICATIONS-----------
\section{Publications}
    \vspace{-5pt}
    \resumeSubHeadingListStart

      \resumeProjectHeading
          {\textbf{INTERCON 2022} $|$ \emph{IEEE}}{September 2022}
          \resumeItemListStart
            \resumeItem{A. Martinez, F. Berrospi, V. Porras and M. Portocarrero, "Using facial landmarks to detect driver fatigue," 2022 IEEE XXIX International Conference on Electronics, Electrical Engineering and Computing (INTERCON), 2022, pp. 1-4, doi: 10.1109/ INTERCON55795.2022.9870046.}
          \resumeItemListEnd

    \resumeSubHeadingListEnd
\vspace{-15pt}

%-----------EDUCATION-----------
\section{Education}
  \resumeSubHeadingListStart
    \resumeSubheading
      {Purdue University}{December 2019}
      {BS in Industrial Engineering}{West Lafayette, IN}
  \resumeSubHeadingListEnd


\end{document}
